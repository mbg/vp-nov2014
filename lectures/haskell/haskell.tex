% use aspectratio=169 for 16:9 aspect ratio (widescreen)
% use aspectratio=1610 for 16:10 aspect ratio (widescreen)
% use aspectratio=43 for 4:3 aspect ratio 
\documentclass[aspectratio=43]{beamer}
\usepackage{fancyeq}


\usetheme{uoc}

% change this to 'uoc-blue', 'uoc-red', etc. for different
% colour themes
\usecolortheme{uoc-purple}

% comment out if you do not want serif fonts
\usefonttheme{serif}

% comment out if you want the navigation bar
\beamertemplatenavigationsymbolsempty

% change for conference or comment out to use 'University of Cambridge'
\renewcommand{\conference}{Conference'14}

\title{Programming in Haskell}
\subtitle{Villiers Park 2014}
\date{18 November 2014}
\author{Michael B. Gale}
\institute{Computer Laboratory, University of Cambridge}

\newcommand{\keyword}[1]{{\color{blue}#1}}

\newcommand{\expressions}[1]{
\begin{tikzpicture}[remember picture,overlay]
\node [shift={(-4 cm,-1.5cm)}] at (current page.north east)
{
\begin{tikzpicture}[remember picture, overlay]
\usebeamercolor{structure dark};
\node[fill=bg,text=fg,font=\tiny,anchor=north west,minimum width=3.5cm, align=left, text width=2.9cm] at (-0.01cm,0cm){\\#1};
\usebeamercolor{structure light};
\fill[bg] (0,0) rectangle(3.5cm,-0.1cm);
\end{tikzpicture}
};
\end{tikzpicture}
}

\begin{document}
\begin{frame}
\maketitle
\end{frame}

\begin{frame}{Computers are dumb...}
We need to give them step-by-step instructions on how to do things:\\[1cm]
\begin{tabular}[t]{l}
\texttt{\keyword{int} n = 0;} \\
\texttt{\keyword{for}(\keyword{int} i=0; i<20; i++)} \\
\texttt{\{} \\
\texttt{   n += i;}\\
\texttt{\}}
\end{tabular} 
\end{frame}

\begin{frame}{Specfication}
But we just say ``a program that sums all numbers from 0 to 20'' or give a formal, mathematical specification:
\begin{center}
\begin{displaymath}
\sum_{i=0}^{20}
\end{displaymath}
\end{center}
Formal specification is particularly useful as we can use proof techniques to reason about it.
\end{frame}

\begin{frame}
Specification and implementation are disjoint, despite representing the same thing:
\begin{center}
\begin{tabular}{cc}
\textbf{Specification} &  \textbf{Implementation} \\ \hline \\[-0.3cm]
\begin{tabular}{@{}l@{}}
$\displaystyle\sum_{i=0}^{20} i$ 
\end{tabular}&  \begin{tabular}[t]{@{}l@{}}
\texttt{\keyword{int} n = 0;} \\
\texttt{\keyword{for}(\keyword{int} i=0; i<20; i++)} \\
\texttt{\{} \\
\texttt{   n += i;}\\
\texttt{\}}
\end{tabular} 
\end{tabular}
\end{center}
Either one could contain errors!
\end{frame}

\begin{frame}{Functional Programming}
Programming paradigm based on mathematical functions rather than procedures:
\begin{center}
\texttt{sum [0..20]}
\end{center}
Specification is the implementation! The compiler works out how to translate this into something a computer can understand.
\end{frame}

\begin{frame}
Imperative programming is based on statements which manipulate values in memory. \\[1cm]
Functional programming is based on the evaluation of expressions.
\end{frame}

\begin{frame}{Haskell}
A non-strict, purely-functional programming language.
\begin{itemize}
\item Non-strict: only evaluates expressions when their value is needed
\item Purely-functional: the result of a function can only depend on the arguments
\end{itemize}
\end{frame}

\begin{frame}{Expressions}
\expressions{
\textbf{Expressions} \\
Values: \texttt{6}, \texttt{"Hello"}, $\ldots$ \\ Operators: \texttt{e1 + e2}}
\end{frame}

\end{document}

